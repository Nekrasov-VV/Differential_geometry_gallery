\documentclass[10pt]{beamer}
\usepackage[T2A]{fontenc}
\usepackage[utf8]{inputenc}
\usepackage[english,russian]{babel}
\usepackage{pgf}
\usepackage{import}
% Данные для заголовочного слайда
\title{Галерея дифференциальной геометрии}
\institute{Механико-математический факультет МГУ}
\author{В.~В.~Некрасов}
\date{2020}
\usefonttheme[onlymath]{serif}
\usecolortheme{crane}
\bibliographystyle{plainnat}

\begin{document}

\begin{frame}
\titlepage
\end{frame}

\begin{frame}{О чём эта презентация?}
Представлены изображения кривых, поверхностей и других геометрических объектов, с которыми мне удалось встретиться в курсе дифференциальной геометрии.

Большая часть номеров --- это домашние задачи из сбоника \cite{1}.

Изображения получены с помощью python-пакета matplotlib.
\end{frame}

\begin{frame}{№2.18 \cite{1}}
  \[y^2+x^2=8x, y^2=\frac{x^3}{2-x}\]
	\centering

	\fbox{\import{Images/Curves}{2_18.pgf} }
\end{frame}

\begin{frame}{№2.19 \cite{1}}
  \[x^2=4y, y=\frac{8}{x^2+4}\]
	\centering

	\fbox{\import{Images/Curves}{2_19.pgf} }
\end{frame}

\begin{frame}{Список используемой литературы}
\begin{thebibliography}{3} 
\bibitem{1} МИЩЕНКО А. С, СОЛОВЬЕВ Ю. П., ФОМЕНКО А. Т. Сборник задач по дифференциальной геометрии и топологии: Учеб. пособие для вузов.—2-е изд., перераб. и доп.—М.: Издательство физико-математическойлитературы, 2004.—412 с—ISBN 5-94052-078-2.
\end{thebibliography}

\end{frame}

\end{document}
